% -*- Mode:TeX -*-

%% IMPORTANT: The official thesis specifications are available at:
%%            http://libraries.mit.edu/archives/thesis-specs/
%%
%%            Please verify your thesis' formatting and copyright
%%            assignment before submission.  If you notice any
%%            discrepancies between these templates and the 
%%            MIT Libraries' specs, please let us know
%%            by e-mailing thesis@mit.edu

%% The documentclass options along with the pagestyle can be used to generate
%% a technical report, a draft copy, or a regular thesis.  You may need to
%% re-specify the pagestyle after you \include  cover.tex.  For more
%% information, see the first few lines of mitthesis.cls. 

%\documentclass[12pt,vi,twoside]{mitthesis}
%%
%%  If you want your thesis copyright to you instead of MIT, use the
%%  ``vi'' option, as above.
%%
%\documentclass[12pt,twoside,leftblank]{mitthesis}
%%
%% If you want blank pages before new chapters to be labelled ``This
%% Page Intentionally Left Blank'', use the ``leftblank'' option, as
%% above. 

\documentclass[12pt,twoside]{mitthesis}
\usepackage{lgrind}
%% These have been added at the request of the MIT Libraries, because
%% some PDF conversions mess up the ligatures.  -LB, 1/22/2014
\usepackage{cmap}
\usepackage[T1]{fontenc}
\pagestyle{plain}
\usepackage[utf8]{inputenc}
\usepackage[spanish]{babel}
\selectlanguage{spanish}

%% This bit allows you to either specify only the files which you wish to
%% process, or `all' to process all files which you \include.
%% Krishna Sethuraman (1990).

%\include{all}
%% [\files]{Enter file names to process, (chap1,chap2 ...), or `all' to process all files:}
\def\all{all}


\begin{document}

% -*-latex-*-
% 
% For questions, comments, concerns or complaints:
% thesis@mit.edu
% 
%
% $Log: cover.tex,v $
% Revision 1.8  2008/05/13 15:02:15  jdreed
% Degree month is June, not May.  Added note about prevdegrees.
% Arthur Smith's title updated
%
% Revision 1.7  2001/02/08 18:53:16  boojum
% changed some \newpages to \cleardoublepages
%
% Revision 1.6  1999/10/21 14:49:31  boojum
% changed comment referring to documentstyle
%
% Revision 1.5  1999/10/21 14:39:04  boojum
% *** empty log message ***
%
% Revision 1.4  1997/04/18  17:54:10  othomas
% added page numbers on abstract and cover, and made 1 abstract
% page the default rather than 2.  (anne hunter tells me this
% is the new institute standard.)
%
% Revision 1.4  1997/04/18  17:54:10  othomas
% added page numbers on abstract and cover, and made 1 abstract
% page the default rather than 2.  (anne hunter tells me this
% is the new institute standard.)
%
% Revision 1.3  93/05/17  17:06:29  starflt
% Added acknowledgements section (suggested by tompalka)
% 
% Revision 1.2  92/04/22  13:13:13  epeisach
% Fixes for 1991 course 6 requirements
% Phrase "and to grant others the right to do so" has been added to 
% permission clause
% Second copy of abstract is not counted as separate pages so numbering works
% out
% 
% Revision 1.1  92/04/22  13:08:20  epeisach

% NOTE:
% These templates make an effort to conform to the MIT Thesis specifications,
% however the specifications can change.  We recommend that you verify the
% layout of your title page with your thesis advisor and/or the MIT 
% Libraries before printing your final copy.
\title{Aplicación de Slow Education en las Universidades}

\author{Ivan Hansen}
% If you wish to list your previous degrees on the cover page, use the 
% previous degrees command:
%       \prevdegrees{A.A., Harvard University (1985)}
% You can use the \\ command to list multiple previous degrees
%       \prevdegrees{B.S., University of California (1978) \\
%                    S.M., Massachusetts Institute of Technology (1981)}
\department{Departamento de Posgrado}

% If the thesis is for two degrees simultaneously, list them both
% separated by \and like this:
% \degree{Doctor of Philosophy \and Master of Science}
\degree{Tesis de Especialización en Ing en Sistemas de informatión}

% As of the 2007-08 academic year, valid degree months are September, 
% February, or June.  The default is June.
\degreemonth{Sep}
\degreeyear{2015}
\thesisdate{Agosto 28, 2015}

%% By default, the thesis will be copyrighted to MIT.  If you need to copyright
%% the thesis to yourself, just specify the `vi' documentclass option.  If for
%% some reason you want to exactly specify the copyright notice text, you can
%% use the \copyrightnoticetext command.  
%\copyrightnoticetext{\copyright IBM, 1990.  Do not open till Xmas.}

% If there is more than one supervisor, use the \supervisor command
% once for each.
\supervisor{Tutor}{Associate Professor}

% This is the department committee chairman, not the thesis committee
% chairman.  You should replace this with your Department's Committee
% Chairman.
\chairman{Chairman}{Chairman, Department Committee on Graduate Theses}

% Make the titlepage based on the above information.  If you need
% something special and can't use the standard form, you can specify
% the exact text of the titlepage yourself.  Put it in a titlepage
% environment and leave blank lines where you want vertical space.
% The spaces will be adjusted to fill the entire page.  The dotted
% lines for the signatures are made with the \signature command.
\maketitle

% The abstractpage environment sets up everything on the page except
% the text itself.  The title and other header material are put at the
% top of the page, and the supervisors are listed at the bottom.  A
% new page is begun both before and after.  Of course, an abstract may
% be more than one page itself.  If you need more control over the
% format of the page, you can use the abstract environment, which puts
% the word "Abstract" at the beginning and single spaces its text.

%% You can either \input (*not* \include) your abstract file, or you can put
%% the text of the abstract directly between the \begin{abstractpage} and
%% \end{abstractpage} commands.

% First copy: start a new page, and save the page number.
\cleardoublepage
% Uncomment the next line if you do NOT want a page number on your
% abstract and acknowledgments pages.
% \pagestyle{empty}
\setcounter{savepage}{\thepage}
\begin{abstractpage}
% $Log: abstract.tex,v $
% Revision 1.1  93/05/14  14:56:25  starflt
% Initial revision
% 
% Revision 1.1  90/05/04  10:41:01  lwvanels
% Initial revision
% 
%
%% The text of your abstract and nothing else (other than comments) goes here.
%% It will be single-spaced and the rest of the text that is supposed to go on
%% the abstract page will be generated by the abstractpage environment.  This
%% file should be \input (not \include 'd) from cover.tex.
El movimiento Slow se origin\'o a mediados de 1980 en Italia. Actualmente posee m\'ultiples y variadas aplicaciones, SlowEducation es una de ellas.\\
La investigaci\'on y desarrollo de esta tesis tiene como objetivo implementar m\'etodos de Slow + SlowEducation a niveles de educaci\'on universitarias, 
con ejemplos pr\'acticos de aplicaci\'on de campo para Ing. en Sistemas de Informaci\'on.\\
Los resultados finales de experimentar m\'etodos Slow en educaci\'on son los mismos, varian las carreras de grado dependiendo de herramientas 
y tecnolog\'ias aplicadas.\\
La educaci\'on lenta una relaci\'on entre el estudiante, el profesor y la Universidad, conformando as\'i entre estos tres elementos una sociedad de 
aprendizaje. El objetivo de aplicar estos m\'etodos est\'a orientado a afianzar las competencias b\'asicas adquiridas por el alumnado y la transferibilidad 
de \'estas a la vida cotidiana profesional/laboral.



\end{abstractpage}

% Additional copy: start a new page, and reset the page number.  This way,
% the second copy of the abstract is not counted as separate pages.
% Uncomment the next 6 lines if you need two copies of the abstract
% page.
% \setcounter{page}{\thesavepage}
% \begin{abstractpage}
% % $Log: abstract.tex,v $
% Revision 1.1  93/05/14  14:56:25  starflt
% Initial revision
% 
% Revision 1.1  90/05/04  10:41:01  lwvanels
% Initial revision
% 
%
%% The text of your abstract and nothing else (other than comments) goes here.
%% It will be single-spaced and the rest of the text that is supposed to go on
%% the abstract page will be generated by the abstractpage environment.  This
%% file should be \input (not \include 'd) from cover.tex.
El movimiento Slow se origin\'o a mediados de 1980 en Italia. Actualmente posee m\'ultiples y variadas aplicaciones, SlowEducation es una de ellas.\\
La investigaci\'on y desarrollo de esta tesis tiene como objetivo implementar m\'etodos de Slow + SlowEducation a niveles de educaci\'on universitarias, 
con ejemplos pr\'acticos de aplicaci\'on de campo para Ing. en Sistemas de Informaci\'on.\\
Los resultados finales de experimentar m\'etodos Slow en educaci\'on son los mismos, varian las carreras de grado dependiendo de herramientas 
y tecnolog\'ias aplicadas.\\
La educaci\'on lenta una relaci\'on entre el estudiante, el profesor y la Universidad, conformando as\'i entre estos tres elementos una sociedad de 
aprendizaje. El objetivo de aplicar estos m\'etodos est\'a orientado a afianzar las competencias b\'asicas adquiridas por el alumnado y la transferibilidad 
de \'estas a la vida cotidiana profesional/laboral.



% \end{abstractpage}

\cleardoublepage

\section*{Acknowledgments}

Los conocimientos para el desarrollo de este trabajo:
\begin{itemize}
 \item Egresado de la Universidad Tecnológica Nacional - Facultad Regional Tucumán.
 \item Practicante del movimiento Slow.
 \item Entusiasta innovador.
\end{itemize}


%%%%%%%%%%%%%%%%%%%%%%%%%%%%%%%%%%%%%%%%%%%%%%%%%%%%%%%%%%%%%%%%%%%%%%
% -*-latex-*-

% Some departments (e.g. 5) require an additional signature page.  See
% signature.tex for more information and uncomment the following line if
% applicable.
% % -*- Mode:TeX -*-
%
% Some departments (e.g. Chemistry) require an additional cover page
% with signatures of the thesis committee.  Please check with your
% thesis advisor or other appropriate person to determine if such a 
% page is required for your thesis.  
%
% If you choose not to use the "titlepage" environment, a \newpage
% commands, and several \vspace{\fill} commands may be necessary to
% achieve the required spacing.  The \signature command is defined in
% the "mitthesis" class
%
% The following sample appears courtesy of Ben Kaduk <kaduk@mit.edu> and
% was used in his June 2012 doctoral thesis in Chemistry. 

\begin{titlepage}
\begin{large}
This doctoral thesis has been examined by a Committee of the Department
of Chemistry as follows:

\signature{Professor Jianshu Cao}{Chairman, Thesis Committee \\
   Professor of Chemistry}

\signature{Professor Troy Van Voorhis}{Thesis Supervisor \\
   Associate Professor of Chemistry}

\signature{Professor Robert W. Field}{Member, Thesis Committee \\
   Haslam and Dewey Professor of Chemistry}
\end{large}
\end{titlepage}


\pagestyle{plain}
  % -*- Mode:TeX -*-
%% This file simply contains the commands that actually generate the table of
%% contents and lists of figures and tables.  You can omit any or all of
%% these files by simply taking out the appropriate command.  For more
%% information on these files, see appendix C.3.3 of the LaTeX manual. 
\tableofcontents
\newpage
\listoffigures
\newpage
\listoftables


%% This is an example first chapter.  You should put chapter/appendix that you
%% write into a separate file, and add a line \include{yourfilename} to
%% main.tex, where `yourfilename.tex' is the name of the chapter/appendix file.
%% You can process specific files by typing their names in at the 
%% \files=
%% prompt when you run the file main.tex through LaTeX.

\chapter{Introducci\'on}
%Marcar esto como referencia
\textit{Las escuelas lentas lentas proporcionan el descubrimiento del gusto por el saber, mientras que las r\'apidas dan siempre las mismas hambuerguesas (Maurice Holt, 
1992)\\}
En Argentina, los ritmos educativos se aceleran a medida que los alumnos cursan sus niveles. Si comparamos la velocidad de ense\~nanza/aprendizaje entre la escuela 
primaria y secundaria notaremos significantes resultados, sin mencionar las diferencias entre Universidades y niveles secundarios.\\
¿A qu\'e hacemos referencia cuando hablamos de \textit{velocidad de aprendizaje}? o ¿ Qu\'e es la \textit{velocidad de ense\~nanza}?. La educaci\'on como 
actividad es una relaci\'on entre alumnos y docentes, como en toda relaci\'on hay que unificar medidas, metas, ritmos. En esta relaci\'on el docente impone 
m\'etodos de ense\~nanza, muchas veces tambi\'en descubren nuevas cosas, y los alumnos aprenden, la unidad que fuciona y pone a girar estos engrenajes es la llamada velocidad.\\
Se tiene por mal sentado que la mayor velocidad es sin\'onimo de bueno o mejor, en educaci\'on sobran ejemplos que es todo lo contrario. Nuestro sistema 
educativo permite cursar un secundario justamente llamado \textbf{acelerado} donde los resultado finales no son para nada comparables contra los secundarios  de
tiempo tradicional. El cursado de una materia en cualquier nivel educativo es distinto cuando su dictado dura cinco en lugar de diez meses.\\
El desenlace de este trabajo de investigaci\'on gira en torno a como mejorar esa velocidad de aprendizaje y ense\~nanza, como lograr una prudente desaceleraci\'on
de esas unidades.\\
Los ritmos de aprendizaje universitarios pueden tener vinculaci\'on con los siguientes factores: edad, madurez psicol\'ogica, motivaci\'on, preparaci\'on 
previa, dominio cognitivo de estrategias, uso de inteligencias m\'ultiples, etc. Cada desarrollo de los cap\'itulos venideros presenta una herramienta que 
ayuda a potenciar los ritmos de aprendizaje y velocidad de ense\~nanza.


\section{Hablemos de Slow}
Cuando hablamos de Slow las primeras palabras que aparecen para hacer referencia a esta terminolog\'ia son:
\begin{itemize}
 \item Tiempo.
 \item \textit{Slow Food}.
 \item Movimiento Slow.
 \item Ideolog\'ias.
 \item Estilos de vida.
 \item Filosof\'ia.
\end{itemize}
Ahora bien, cuesta entender tantas definiciones alrededor de la palabra \textbf{Slow} y su aplicaci\'on con educaci\'on universitaria. Justamente para tener m\'as
 contexto sobre el tema central, vamos al detalle. \\
\textit{En la vida hay algo m\'as importante que incrementar su velocidad. --Gandhi}\\
El manifiesto Slow cita:\\
Hoy m\'as que nunca, el individuo moderno vive sumido en una particular carrera de obst\'aculos en la que controlar el cron\'ometro hasta la mil\'esima determina
nuestra existencia. La desconexi\'on del medio natural y su tempo, ligado a las estaciones y dem\'as factores que escapan a nuestro control, parece un espejismo
en las sociedades occidentales de hoy en d\'ia. Las ciudades se vuelven an\'onimas y levitamos en nuestro universo de intereses. La prisa es el motor de todas 
nuestras acciones y la cin\'etica de correr envuelve nuestra vida aceler\'andola, economizando cada segundo, rindiendo culto a una velocidad que no nos hace ser 
mejores.\\
El movimiento Slow no pretende abatir los cimientos de lo construido hasta la fecha. Su intenci\'on es iluminar la posibilidad de llevar una vida m\'as plena y 
desacelerada, haciendo que cada individuo pueda controlar y adue\~narse de su propio periplo vital. La clave reside en un juicio acertado de la marcha adecuada 
para cada momento de la carrera diaria.


\subsection{Movimiento Slow}
El movimiento Slow tiene su g\'enesis en la Plaza de Espa\~na romana, en el año 1986. Su nacimiento es indisociable de cierta actitud contestataria en clara 
oposici\'on a la americanizaci\'on de Europa. Cuando el periodista Carlo Petrini se top\'o con la apertura de un conocido establecimiento de comida r\'apida en 
este enclave hist\'orico de la capital italiana, algo se removi\'o en su interior. Definitivamente, se hab\'ian traspasado los l\'imites de lo aceptable y 
entendi\'o, de forma casi visionaria, los peligros que se cern\'ian sobre los h\'abitos alimentarios de la poblaci\'on del viejo continente, ofuscado en imitar 
el tempo vital marcado al otro lado del Atl\'antico. La respuesta no se hizo esperar, fund\'andose la semilla del movimiento; \textbf{Slow Food}.\\
La idea era simple; proteger los productos estacionales, frescos y aut\'octonos del acoso de la comida r\'apida y defender los intereses de los productos locales,
siempre en un r\'egimen sostenible, a trav\'es del culto a la diversidad, alertando de los peligros evidentes de la explotaci\'on intensiva de la tierra con 
fines comerciales.\\
Tras Slow Food, aparecer\'ian nuevas aplicaciones a otros \'ambitos esenciales de nuestras existencias como el sexo, la salud, el trabajo, la educación o el ocio
que acabar\'ian por conformar las \'areas de influencia del movimiento Slow. 


\subsection{Slow Education}
Entre las comunidades de Slow se refer\'ian a Slow Education o Slow Schooling a aquellas entidades donde en los comedores escolares serv\'ian o promov\'ian Slow Food.
Este concepto cambi\'o radicalmente cuando los m\'etodos te\'oricos/pr\'acticos de Slow se aplicaron de lleno a la ense\~nanza y aprendizaje, es decir, al sistema
educativo.\\
El campo de aplicaci\'on de Slow Education est\'a enfocado en las escuelas primarias donde el principal tema de trabajo es el ni\~no y todo lo que lo rodea. Su 
ambiente de crecimiento, su predisposici\'on al aprendizaje, su alimentaci\'on, sus recreos, su integraci\'on y por sobre todo, la forma de aprender. Se aspira 
a tener como resultado ni\~nos felices, que disfruten mientras aprenden y crescan, calidad educativa y vida sana.\\
El panorama en la educaci\'on universitaria es completamente distinto porque ya no tratamos con ni\~nos, los ritmos y responsabilidades de un estudiante a ser 
formado en una carrera de grado son otras. El desaf\'io est\'a en mantener la calidad educativa que Slow Education propone y fomentar sus pautas en herramientas
pr\'acticas.

\subsection{Elogio de la educaci\'on lenta}
\'Este t\'itulo hace referencia a un libro publicado en el a\~no 2010, cuyo autor \textit{Joan Dom\'enech Francesch} refleja las pautas de la lentitud en los 
sistemas educativos y su evoluci\'on marcada por resultados positivos.\\
Cuando hablamos sobre sistemas educativos es imposible escapar relacionar el \'ambito social al que educadores y alumnos est\'an sometidos. Mi reflexi\'on a 
\'esta actualidad Argentina-Tucumana es sobre la oferta/demanda en Ingenieros en Sistemas graduados para saciar la sed de empresas globales, este es uno de los factores
que obligan a las Universidades a reinventarse y acelerar los procesos educacionales a cualquier costo descuidando la calidad del graduado, programas sobrecargados
y objetivos b\'asicos a ser alcanzados antes de tiempo.\\
Muchos efectos negativos que el sistema educativo actual posee son por una concepci\'on erronea y un tratamiento equivocado del tiempo. Sensaci\'on constante
de falta de tiempo del profesor y alumno, dificultades para desarrollar diversidad, aprendizajes desfasados, horarios y programas sobrecargados, desvinculaci\'on
entre el pasado y el presente, son algunos de los ejemplos a atacar aplicando la desaceleraci\'on en la educaci\'on moderna y Slow.\\

Este trabajo no propone ning\'un plan 360 de cambio en el sistema educativo, simplemente una mirada diferente al tiempo correcto de educar y un replanteaminto 
individual y colectivo, que afecte al marco social, universitario y familiar.

\section{Estado actual}\label{ch1:opts}
\section{Objetivos y aplicaci\'on}
Los movimientos Slow coinciden en:
\begin{itemize}
 \item Buscar el tiempo justo.
 \item Insistir en la calidad.
 \item Devolver el tiempo de creaci\'on a las personas.
 \item Trabajar en el presente, bas\'andose en el pasado y pensando en el futuro.
 \item Posicionar de manera cr\'itica el estado actual.
\end{itemize}
Como ingenieros debemos ser pr\'acticos, entonces como objetivos no podemos ser menos que realistas. En los siguientes cap\'itulos desarrollaremos cuantro m\'etodos
sobre como aplicar los \'items antes mencionados. Cada uno se transformar\'a en una herramienta independiente para poder ser puesta en marcha con cualquier 
programa o curr\'icula de materias en la carrera de grado de Ing en sistemas de informaci\'on.\\


\section{Los beneficios de aplicar este trabajo}
En primer lugar, es fundamental analizar el ritmo de aprendizaje del alumnado.

%% This is an example first chapter.  You should put chapter/appendix that you
%% write into a separate file, and add a line \include{yourfilename} to
%% main.tex, where `yourfilename.tex' is the name of the chapter/appendix file.
%% You can process specific files by typing their names in at the 
%% \files=
%% prompt when you run the file main.tex through LaTeX.
\chapter{M\'etodo progresivo-incremental}
Fomentar el aprendizaje colaborativo: el alumnado trabaja en grupo interactuando, ayud\'andose y favoreciendo la comunicaci\'on.\\


Trabajar la creatividad y el descubrimiento: el proceso debería ser tan importante como el resultado.\\
Argumentar, reflexionar, escuchar, debatir: son aspectos fundamentales para adquirir conocimientos. Resulta muy útil buscar actividades que permitan 
trabajar estos aspectos.\\
No penalizar el error: Joan Domènech en su libro  "Elogio de la Educación Lenta"1 considera positivo organizar actividades que permitan al alumnado asumir 
el error como parte del proceso de aprendizaje.



\section{Definici\'on del m\'etodo}

\subsection{¿Para que sirve? ¿C\'omo se usa?}
\subsection{Evaluaci\'on}
\subsection{Claves}
Dar tiempo y espacio a la creatividad\\
Seleccionando un tema libremente se it emphasizes student interests

\subsection{Beneficios}

\section{Herramienta}\label{ch1:opts}

\section{Resultados}


%% This is an example first chapter.  You should put chapter/appendix that you
%% write into a separate file, and add a line \include{yourfilename} to
%% main.tex, where `yourfilename.tex' is the name of the chapter/appendix file.
%% You can process specific files by typing their names in at the 
%% \files=
%% prompt when you run the file main.tex through LaTeX.
\chapter{M\'etodo progresivo-incremental}
\emph{
  \begin{center}Fomentar el aprendizaje colaborativo: el alumnado trabaja en grupo interactuando, ayud\'andose, favoreciendo la comunicaci\'on y as\'i 
  adequiriendo conocimientos entre todos.
  \end{center}}
Trabajar la creatividad y el descubrimiento: el proceso debería ser tan importante como el resultado.\\
Argumentar, reflexionar, escuchar, debatir: son aspectos fundamentales para adquirir conocimientos. Resulta muy útil buscar actividades que permitan 
trabajar estos aspectos. No penalizar el error: Joan Domènech Francesch en su libro ``Elogio de la Educaci\'on Lenta''\cite{Joan} considera positivo organizar 
actividades que permitan al alumnado asumir el error como parte del proceso de aprendizaje.

\section{Desarrollo del m\'etodo}
En los procesos tradicionales de educaci\'on los docentes suelen desarrollar un tema, ponerlo en pr\'actica y evaluarlo. \'Este m\'etodo \textit{Slow}
est\'a enfocado en aspectos diferentes, prioriza la participaci\'on, la cooperaci\'on e independencia en el aprendizaje.\\
Vamos a dise\~nar la siguiente receta para poner este m\'etodo en pr\'actica:
\begin{enumerate}
 \item Selecionar un tema del contenio sin estimar su tiempo de culminaci\'on.
 \item Formar grupos entre los alumnos.
 \item Consensuar una plataforma com\'un donde desarrollar los temas en cuesti\'on.
 \item Planificar etapas y entregas.
 \item Evaluaci\'on.
\end{enumerate}
Cada uno de estos puntos pueden ser resueltos el primer d\'ia de clase. Detallaremos a continuaci\'on cada uno de ellos.
\begin{enumerate}
 \item \textbf{Selecionar un tema del contenio sin estimar su tiempo de culminaci\'on:}\\
 Como primer paso de esta herramienta es importante destacar algo; el tema que vamos a elegir no necesariamente tiene que ser parte de la curr\'icula pero, tener 
 en cuenta que su desarrollo preferentemente durar\'a lo que dura el dictado de la materia, o m\'as, o no necesariamente termine (puede ser un caso para 
 investigaciones futuras).\\
 Cuanto m\'as troncal sea el tema, m\'as provechoso ser\'a el aprendizaje.\\ 
 Podemos seleccionar dos o tres temas que est\'en relacionados entre si, por ejemplo, si el objetivo es que los alumnos aprendan las operaciones matem\'aticas, 
 podemos incluir como temas a desarrollar: suma, resta, multiplicaci\'on y divisi\'on e iterar con cada uno de ellos.
 
 \item \textbf{Formar grupos entre los alumnos:}\\
 La din\'amica de formar grupos entre los estudiantes es para trabajar en forma independiente donde cada grupo tendr\'a su objetivo: el desarrollo de uno de los
 temas escogidos. El trabajo es en grupos y a su vez se necesita de todos (equipos multidisciplinarios) para completar los dem\'as temas. Cada grupo contar\'a 
 con un tema particular, siguiendo el ejemplo de las operaciones matem\'aticas, un grupo trabajar\'a sobre sumas, otro sobre restas y as\'i podemos dividir los 
 temas en cuanto grupos tengamos. La clave en el transcurso de este mecan\'ismo es poder hacer que los distintos grupos interactuen uno entre con otros, por 
 ejemplo, la gente que trabaja en \textit{multiplicaci\'on}, necesitar\'a material del grupo de \textit{suma}.
 
 \item \textbf{Consensuar una plataforma com\'un donde desarrollar los temas en cuesti\'on:}\\
 \'Esta plataforma tiene que ser de f\'acil acceso, configurable a las necesidades de los usuarios, sostenible en el transcurso del tiempo y sobre 
 todo visible, disponible para todos.\\
 Puede ser un documento compartido, un blog en internet, entre otros. Voy a mostrarles una poderosa herramienta para este caso: \emph{gitbook}\cite{gitbook}
 Gitbook es una moderna plataforma coolaborativa en la cual podemos escribir, desarrollar y hacer entregas tempranas sobre nuestro trabajo. Solo necesitamos 
 crear usuarios para nuestros estudiantes y seguir los instructivos.
 
 \item \textbf{Planificar etapas y entregas:}\\
 Cuando ya tenemos elegido el o los temas a desarrollar y definido los equipos de trabajo, es necesario planificar la forma de entrega del trabajo. En la primera
 interacci\'on casi siempre resulta un experimento para los alumnos, por lo que los resultados tienden a fallar. Es justamente esto lo que buscamos, entonces 
 para las pr\'oximas entregas tenemos que aprender de los errores cometidos, corregirlos y mejorar.\\
 Hablamos de planificar y concordar entregas entre los grupos de trabajo y el docente, esto significa que los desarrollos tienen que estar divididos en entapas, 
 etapas entregables las cuales tambi\'en puedan ser evaluadas.
 
 \item \textbf{Evaluaci\'on:}\\
 El uso de este tipo de plataformas coolaborativas nos permiten evaluar constantemente el trabajo de cada grupo como as\'i tambien cooperar aportando conocimientos
 y mejorando los contenidos. Entonces, para la entrega final el contenido es el resultado de las sumas de las partes, por lo tanto la evaluaci\'on es grupal
 pero puede estar dada individualmente tambi\'en si el docente lo requiere. Podemos evaluar en cada entrega, tambi\'en podemos evaluar la calidad de entrega de 
 cada grupo y su predisposici\'on a ayudar a los dem\'as.\\
 Al inicio, en la selecci\'on del tema debemos clarificar el resultado esperado.
 \end{enumerate}

\subsection{github}
\title{Licencias}
Apache License 2.0 
GNU General Public License v2.0
MIT License
Artistic License 2.0
BSD 2-clause "Simplefied" License
BSD 3-clause "New" or "Revised" License
Creative Commons Zero v1.0 Universal
Eclipse Public License 1.0
GNU Alfero General Public v3.0
GNU General Public License v3.0
GNU Lesser General Public License v2.1
GNU Lesser General Public License v3.0
ISC License
Mozilla Public License 2.0
The Unlicense
 
\section{Claves}
\begin{center}
 \textit{ Donde, cuando y como interviene el docente?}
\end{center} 
El docente interviene durante todo el per\'iodo de desarrollo. Estas son las actividades propuestas:\\
\begin{enumerate}
 \item Monitoreo constante sobre el desarrollo. Sobre todo controlar si las bases y contenidos de las investigaciones realizadas por los grupos de alumnos son
 correctas.
 \item Proponer debates, ampliar contenidos, ejemplificar, son tareas en las que el docente puede intervenir.
 \item Resolver y evacuar dudas que en oportunidades los grupos pueden pedir apoyo al docente.
 \item Evaluaciones parciales sobre entregas o demostraciones, esto permite tener un mejor entendimiento y llegar a una conclusi\'on bien formada al final del
 trabajo.
\end{enumerate}

\section{Beneficios}
Podemos listar los siguientes beneficios:\\
\begin{itemize}
 \item Participaci\'on de cada uno de los alumnos.
 \item Dar tiempo y espacio a la creatividad.
 \item Cooperaci\'on entre los equipos.
 \item Compromiso entre todos para lograr un resultado com\'un.
 \item Fomentar la libre expresi\'on y libertad a la hora de elegir como llevar a cabo el trabajo.
 \item Evaluaci\'on cruzada entre los integrantes de los equipos.
 \item Correcciones tempranas.
\end{itemize}

\section{Conclusi\'on y Resultados}
La aplicaci\'on de esta herramienta o m\'etodo de ense\~nanza tiene algunas ventajas particulares como; el alumno siente mayor ineter\'es en los temas ya que
encuentra la libertad de elecci\'on a trabajar en sobre algo que \'el prefiere, la cooperaci\'on conjunta, el desarrollo sin tiempos y autoevaluaci\'on son las 
m\'as destacadas.\\
Ahora bien, hablemos de los resultados, el docente debe asumir su rol y liderar las actividades como; revisi\'on de contenidos, comunicaci\'on entre grupos, 
entregas parciales y dem\'as. De este modo llegar de la mejor manera posible al final del trabajo.




\chapter{Aprender a pensar}
\section{Dar tiempo para crear}
\section{a}
\section{b}
\section{Beneficios}
\chapter{Evaluaci\'on Slow}
\section{Una manera diferente de evaluar}
\section{Desarrollo del método}
\section{d}
\section{Beneficios}
\appendix
\chapter{Tables}

\begin{table}
\caption{Armadillos}
\label{arm:table}
\begin{center}
\begin{tabular}{||l|l||}\hline
Armadillos & are \\\hline
our	   & friends \\\hline
\end{tabular}
\end{center}
\end{table}

\clearpage
\newpage

\chapter{Figures}

\vspace*{-3in}

\begin{figure}
\vspace{2.4in}
\caption{Armadillo slaying lawyer.}
\label{arm:fig1}
\end{figure}
\clearpage
\newpage

\begin{figure}
\vspace{2.4in}
\caption{Armadillo eradicating national debt.}
\label{arm:fig2}
\end{figure}
\clearpage
\newpage

%% This defines the bibliography file (main.bib) and the bibliography style.
%% If you want to create a bibliography file by hand, change the contents of
%% this file to a `thebibliography' environment.  For more information 
%% see section 4.3 of the LaTeX manual.
\begin{singlespace}
\bibliography{main}
\bibliographystyle{plain}
\end{singlespace}

\end{document}

