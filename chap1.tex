%% This is an example first chapter.  You should put chapter/appendix that you
%% write into a separate file, and add a line \include{yourfilename} to
%% main.tex, where `yourfilename.tex' is the name of the chapter/appendix file.
%% You can process specific files by typing their names in at the 
%% \files=
%% prompt when you run the file main.tex through LaTeX.

\chapter{Introducci\'on}
%Marcar esto como referencia
\textit{``Las escuelas lentas proporcionan el descubrimiento del gusto por el saber, mientras que las r\'apidas dan siempre las mismas hamburguesas'' 
(Maurice Holt, 1992)\\}
\\
En Argentina, los ritmos educativos se aceleran a medida que los alumnos cursan sus niveles educativos. Si comparamos la velocidad de ense\~nanza/aprendizaje 
entre la escuela primaria y secundaria se evidencian diferencias significativas, sin mencionar las existentes entre nivel secundario y universitario .\\
¿A qu\'e hacemos referencia cuando hablamos de \textit{velocidad de aprendizaje}? o ¿ Qu\'e es la \textit{velocidad de ense\~nanza}?. La educaci\'on como 
actividad es una relaci\'on entre alumnos y docentes, como en toda relaci\'on hay que unificar medidas, metas y ritmos. En esta relaci\'on el docente impone 
m\'etodos de ense\~nanza, los alumnos aprenden y la unidad com\'un que se genera se llama \textit{velocidad}.\\
Se mal interpreta que la mayor velocidad es sin\'onimo de bueno o mejor, sin embargo en educaci\'on sobran ejemplos contrariando esta hip\'otesis. Nuestro sistema
educativo permite cursar un secundario justamente llamado \textbf{acelerado} donde los resultados finales no son para nada equiparables a obtenidos con los 
secundarios de tiempo tradicional. El cursado de una materia en cualquier nivel educativo es distinto cuando su dictado dura cinco en lugar de diez meses.\\
La finalidad de este trabajo de investigaci\'on gira en torno a como mejorar esa velocidad de aprendizaje y ense\~nanza, como lograr una prudente desaceleraci\'on
de esas unidades.\\
Los ritmos de aprendizaje universitarios pueden tener vinculaci\'on con los siguientes factores: edad, madurez psicol\'ogica, motivaci\'on, preparaci\'on 
previa, dominio cognitivo de estrategias, uso de inteligencias m\'ultiples, etc. En el desarrollo de los siguientes cap\'itulos se presentar\'an herramientas que 
ayudan a potenciar los ritmos de aprendizaje y velocidad de ense\~nanza.


\section{Hablemos de Slow}
Cuando hablamos de Slow las primeras palabras que aparecen para hacer referencia a esta terminolog\'ia son:
\begin{itemize}
 \item Tiempo.
 \item \textit{Slow Food}.
 \item Movimiento Slow.
 \item Ideolog\'ias.
 \item Estilos de vida.
 \item Filosof\'ia.
\end{itemize}
Ahora bien, cuesta entender tantas definiciones alrededor de la palabra \textbf{Slow} y su aplicaci\'on con educaci\'on universitaria. Justamente para tener m\'as
 contexto sobre el tema central, vamos al detalle. \\
\textit{``En la vida hay algo m\'as importante que incrementar su velocidad.'' --Gandhi}\\
El manifiesto Slow cita:\\
Hoy m\'as que nunca, el individuo moderno vive sumido en una particular carrera de obst\'aculos en la que controlar el cron\'ometro hasta la mil\'esima, determina
nuestra existencia. La desconexi\'on del medio natural y su tempo, ligado a las estaciones y dem\'as factores que escapan a nuestro control, parece un espejismo
en las sociedades occidentales de hoy en d\'ia. Las ciudades se vuelven an\'onimas y levitamos en nuestro universo de intereses. La prisa es el motor de todas 
nuestras acciones y la cin\'etica de correr envuelve nuestra vida aceler\'andola, economizando cada segundo, rindiendo culto a una velocidad que no nos hace ser 
mejores.\\
El movimiento Slow no pretende abatir los cimientos de lo construido hasta la fecha. Su intenci\'on es iluminar la posibilidad de llevar una vida m\'as plena y 
desacelerada, haciendo que cada individuo pueda controlar y adue\~narse de su propio periplo vital. La clave reside en un juicio acertado de la marcha adecuada 
para cada momento de la carrera diaria.


\subsection{Movimiento Slow}
El movimiento Slow tiene su g\'enesis en la Plaza de Espa\~na romana, en el año 1986. Su nacimiento es indisociable de cierta actitud contestataria en clara 
oposici\'on a la americanizaci\'on de Europa. Cuando el periodista Carlo Petrini se top\'o con la apertura de un conocido establecimiento de comida r\'apida en 
este enclave hist\'orico de la capital italiana, algo se removi\'o en su interior. Definitivamente, se hab\'ian traspasado los l\'imites de lo aceptable y 
entendi\'o, de forma casi visionaria, los peligros que se cern\'ian sobre los h\'abitos alimentarios de la poblaci\'on del viejo continente, ofuscado en imitar 
el tempo vital marcado al otro lado del Atl\'antico. La respuesta no se hizo esperar, fund\'andose la semilla del movimiento; \textbf{Slow Food}.\\
La idea era simple; proteger los productos estacionales, frescos y aut\'octonos del acoso de la comida r\'apida y defender los intereses de los productos locales,
siempre en un r\'egimen sostenible, a trav\'es del culto a la diversidad, alertando de los peligros evidentes de la explotaci\'on intensiva de la tierra con 
fines comerciales.\\
Tras Slow Food, aparecer\'ian nuevas aplicaciones a otros \'ambitos esenciales de nuestras existencias como el sexo, la salud, el trabajo, la educación o el ocio
que acabar\'ian por conformar las \'areas de influencia del movimiento Slow. 


\subsection{Slow Education}
Entre las comunidades de Slow se refer\'ian a Slow Education o Slow Schooling a aquellos establecimientos donde en los comedores escolares serv\'ian o 
promov\'ian Slow Food. Este concepto cambi\'o radicalmente cuando los m\'etodos te\'oricos/pr\'acticos de Slow se aplicaron de lleno a la ense\~nanza y 
aprendizaje, es decir, al sistema educativo.\\
El campo de aplicaci\'on de Slow Education est\'a enfocado en las escuelas primarias donde el principal tema de trabajo es el ni\~no y todo lo que lo rodea. Su 
ambiente de crecimiento, predisposici\'on al aprendizaje,  alimentaci\'on, recreos, integraci\'on y por sobre todo, la forma de aprender. Se aspira 
a tener como resultado calidad educativa, vida sana y ni\~nos felices, que disfruten mientras aprenden y crezcan.\\
El panorama en la educaci\'on universitaria es completamente distinto porque ya no tratamos con ni\~nos, los ritmos y responsabilidades de un estudiante en 
formaci\'on de carrera de grado son diferentes. El desaf\'io est\'a en mantener la calidad educativa que Slow Education propone y fomentar sus pautas en herramientas
pr\'acticas.

\subsection{Elogio de la educaci\'on lenta}
\'Este t\'itulo hace referencia a un libro publicado en el a\~no 2010, cuyo autor \textit{Joan Dom\'enech Francesch} refleja las pautas de la lentitud en los 
sistemas educativos y su evoluci\'on marcada por resultados positivos.\\
Cuando hablamos sobre sistemas educativos es imposible no relacionar el \'ambito social al que educadores y alumnos est\'an sometidos. Reflexionando sobre
la actualidad Argentina-Tucumana en relaci\'on a la sobre demanda en Ingenieros en Sistemas, obligan a las Universidades a reinventarse, acelerando los procesos 
educacionales a cualquier costo, descuidando la calidad en la ense\~nanza del graduado con programas sobrecargados que no siempre llegan a cumplir con los 
objetivos propuestos curricularmente.\\
Muchos efectos negativos que el sistema educativo actual posee son por una concepci\'on err\'onea y un tratamiento equivocado del tiempo. Sensaci\'on constante
de falta de tiempo del profesor y alumno, dificultades para desarrollar diversidad, aprendizajes desfasados, horarios y programas sobrecargados, desvinculaci\'on
entre el pasado y el presente, son algunos de los ejemplos a atacar aplicando la desaceleraci\'on en la educaci\'on moderna y Slow.\\

Este trabajo no propone ning\'un plan de cambio 360 en el sistema educativo, simplemente una mirada diferente al tiempo correcto de educar y un replanteamiento 
individual y colectivo, que afecte al marco social, universitario y familiar.

\section{Estado actual}\label{ch1:opts}
No existen actualmente trabajos formales sobre la aplicaci\'on de t\'ecnicas slow en educaci\'on superior o universitaria, lo que constituye un desaf\'io de este
trabajo.\\
Hoy en d\'ia el movimiento Slow transcurre sobre su m\'axima curva de crecimiento, el fuerte apoyo de miles de comunidades a \textit{Slow City}, las variadas y 
nuevas recetas a nivel mundial para \textit{Slow Food} y las iniciativas de gobiernos Ingleses de incorporar \textit{Slow Education} a sus sistemas educativos
primarios son algunos ejemplos.\\
El desarrollo del movimiento \textit{Slow} llego no tan solo para mejorar la calidad de diversos aspectos, si no tambi\'en, para ser adoptado como estilo de vida.
Entonces, si hablamos puntualmente de la educaci\'on universitaria buscamos aumentar los standares de aprendizaje y fomentar el movimiento \textit{Slow} como una
alternativa de cambio cultural y social.

\section{Objetivos y aplicaci\'on}
Los movimientos \textit{Slow} coinciden en:
\begin{itemize}
 \item Buscar el tiempo justo.
 \item Insistir en la calidad.
 \item Devolver el tiempo de creaci\'on a las personas.
 \item Trabajar en el presente, bas\'andose en el pasado y pensando en el futuro.
 \item Posicionar de manera cr\'itica el estado actual.
\end{itemize}
En los siguientes cap\'itulos desarrollaremos cuatro m\'etodos sobre como aplicar los \'items antes mencionados. Cada uno se transformar\'a en una herramienta 
independiente para poder ser puesta en marcha con cualquier programa o curr\'icula de materias en la carrera de grado de Ingenier\'ia en Sistemas de 
Informaci\'on.\\


\section{Los beneficios de aplicar este trabajo}
Como resumen de este cap\'itulo resaltaremos los beneficios de aplicar las cuatro herramientas desarrolladas.
\begin{enumerate}
 \item Disminuir la velocidad de ense\~nanza ayuda a mejorar la velocidad de aprendizaje.
 \item Fomentar el trabajo social, con feedback cont\'inuo ayuda a retroalimentarse entre pares.
 \item Buscar el tiempo justo. Tiempo para la reflexi\'on, la revisi\'on y la predicci\'on de posibles consecuencias.
 \item Foco en la calidad. Cualquiera de los resultados en la aplicaci\'on de las siguientes herramientas centran sus objetivos en la calidad.
\end{enumerate}
Es muy com\'un escuchar en el mundo de los negocios, \textit{El tiempo es oro}, en la educaci\'on universitaria deberiamos cambiar esa frase por 
\textit{\textbf{El tiempo es vida.}}




