%% This is an example first chapter.  You should put chapter/appendix that you
%% write into a separate file, and add a line \include{yourfilename} to
%% main.tex, where `yourfilename.tex' is the name of the chapter/appendix file.
%% You can process specific files by typing their names in at the 
%% \files=
%% prompt when you run the file main.tex through LaTeX.
\chapter{Introducci\'on}
Las escuelas lentas proporcionan el descubrimiento del gusto por el saber, mientras que las r\'apidas dan siempre las mismas hambuerguesas (Maurice Holt, 1992)
Los ritmos de aprendizaje universitarios pueden tener vinculaci\'on con los siguientes factores: edad, madurez psicol\'ogica, motivaci\'on, preparaci\'on 
previa, dominio cognitivo de estrategias, uso de inteligencias m\'ultiples, etc.


\section{Hablemos de Slow}

\subsection{Movimiento Slow}
\subsection{Slow Education}
\subsection{Slow schooling}



\section{Estado actual}\label{ch1:opts}

\section{Los beneficios de aplicar este trabajo}
En primer lugar, es fundamental analizar el ritmo de aprendizaje del alumnado.
