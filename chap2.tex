%% This is an example first chapter.  You should put chapter/appendix that you
%% write into a separate file, and add a line \include{yourfilename} to
%% main.tex, where `yourfilename.tex' is the name of the chapter/appendix file.
%% You can process specific files by typing their names in at the 
%% \files=
%% prompt when you run the file main.tex through LaTeX.
\chapter{M\'etodo progresivo-incremental}
Fomentar el aprendizaje colaborativo: el alumnado trabaja en grupo interactuando, ayud\'andose y favoreciendo la comunicaci\'on.\\


Trabajar la creatividad y el descubrimiento: el proceso debería ser tan importante como el resultado.\\
Argumentar, reflexionar, escuchar, debatir: son aspectos fundamentales para adquirir conocimientos. Resulta muy útil buscar actividades que permitan 
trabajar estos aspectos.\\
No penalizar el error: Joan Domènech en su libro  "Elogio de la Educación Lenta"1 considera positivo organizar actividades que permitan al alumnado asumir 
el error como parte del proceso de aprendizaje.



\section{Definici\'on del m\'etodo}

\subsection{Para que sirve? C\'omo se usa?}
\subsection{Evaluaci\'on}
\subsection{Claves}
Dar tiempo y espacio a la creatividad\\
Seleccionando un tema libremente se it emphasizes student interests

\subsection{Beneficios}



\section{Herramienta}\label{ch1:opts}

\section{Resultados}

