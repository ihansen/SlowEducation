%% This is an example first chapter.  You should put chapter/appendix that you
%% write into a separate file, and add a line \include{yourfilename} to
%% main.tex, where `yourfilename.tex' is the name of the chapter/appendix file.
%% You can process specific files by typing their names in at the 
%% \files=
%% prompt when you run the file main.tex through LaTeX.
\chapter{M\'etodo progresivo-incremental}
\emph{
  \begin{center}Fomentar el aprendizaje colaborativo: el alumnado trabaja en grupo interactuando, ayud\'andose, favoreciendo la comunicaci\'on y as\'i 
  adequiriendo conocimientos entre todos.
  \end{center}}
Trabajar la creatividad y el descubrimiento: el proceso debería ser tan importante como el resultado.\\
Argumentar, reflexionar, escuchar, debatir: son aspectos fundamentales para adquirir conocimientos. Resulta muy útil buscar actividades que permitan 
trabajar estos aspectos. No penalizar el error: Joan Domènech Francesch en su libro ``Elogio de la Educación Lenta''\cite{Joan} considera positivo organizar actividades que 
permitan al alumnado asumir el error como parte del proceso de aprendizaje.

\section{Desarrollo del m\'etodo}
En los procesos tradicionales de educaci\'on los docentes suelen desarrollar un tema, ponerlo en pr\'actica y evaluarlo. \'Este m\'etodo \textit{Slow}
est\'a enfocado en aspectos diferentes, prioriza la participaci\'on, la cooperaci\'on e independencia en el aprendizaje.\\
Vamos a dise\~nar la siguiente receta para poner este m\'etodo en pr\'actica:
\begin{enumerate}
 \item Selecionar un tema del contenio sin estimar su tiempo de culminaci\'on.
 \item Formar grupos entre los alumnos.
 \item Consensuar una plataforma com\'un donde desarrollar los temas en cuesti\'on.
 \item Planificar etapas y entregas.
 \item Evaluaci\'on.
\end{enumerate}
Cada uno de estos puntos pueden ser resueltos el primer d\'ia de clase. Detallaremos a continuaci\'on cada uno de ellos.
\begin{enumerate}
 \item \textbf{Selecionar un tema del contenio sin estimar su tiempo de culminaci\'on:}\\
 Como primer paso de esta herramienta es importante destacar algo; el tema que vamos a elegir no necesariamente tiene que ser parte de la curr\'icula pero, tener 
 en cuenta que su desarrollo preferentemente durar\'a lo que dura el dictado de la materia, o m\'as, o no necesariamente termine (puede ser un caso para 
 investigaciones futuras).\\
 Cuanto m\'as troncal sea el tema, m\'as provechoso ser\'a el aprendizaje.\\ 
 Podemos seleccionar dos o tres temas que est\'en relacionados entre si, por ejemplo, si el objetivo es que los alumnos aprendan las operaciones matem\'aticas, 
 podemos incluir como temas a desarrollar: suma, resta, multiplicaci\'on y divisi\'on e iterar con cada uno de ellos.
 
 \item \textbf{Formar grupos entre los alumnos:}\\
 La din\'amica de formar grupos entre los estudiantes es para trabajar en forma independiente donde cada grupo tendr\'a su objetivo: el desarrollo de uno de los
 temas escogidos. El trabajo es en grupos y a su vez se necesita de todos (equipos multidisciplinarios) para completar los dem\'as temas. Cada grupo contar\'a 
 con un tema particular, siguiendo el ejemplo de las operaciones matem\'aticas, un grupo trabajar\'a sobre sumas, otro sobre restas y as\'i podemos dividir los 
 temas en cuanto grupos tengamos. La clave en el transcurso de este mecan\'ismo es poder hacer que los distintos grupos interactuen uno entre con otros, por 
 ejemplo, la gente que trabaja en \textit{multiplicaci\'on}, necesitar\'a material del grupo de \textit{suma}.
 
 \item \textbf{Consensuar una plataforma com\'un donde desarrollar los temas en cuesti\'on:}\\
 \'Esta plataforma tiene que ser de f\'acil acceso, configurable a las necesidades de los usuarios, sostenible en el transcurso del tiempo y sobre 
 todo visible, disponible para todos.\\
 Puede ser un documento compartido, un blog en internet, entre otros. Voy a mostrarles una poderosa herramienta para este caso: \emph{gitbook}\cite{gitbook}
 Gitbook es una moderna plataforma coolaborativa en la cual podemos escribir, desarrollar y hacer entregas tempranas sobre nuestro trabajo. Solo necesitamos 
 crear usuarios para nuestros estudiantes y seguir los instructivos.
 
 \item \textbf{Planificar etapas y entregas:}\\
 Cuando ya tenemos elegido el o los temas a desarrollar y definido los equipos de trabajo, es necesario planificar la forma de entrega del trabajo. En la primera
 interacci\'on casi siempre resulta un experimento para los alumnos, por lo que los resultados tienden a fallar. Es justamente esto lo que buscamos, entonces 
 para las pr\'oximas entregas tenemos que aprender de los errores cometidos, corregirlos y mejorar.\\
 Hablamos de planificar y concordar entregas entre los grupos de trabajo y el docente, esto significa que los desarrollos tienen que estar divididos en entapas, 
 etapas entregables las cuales tambi\'en puedan ser evaluadas.
 
 \item \textbf{Evaluaci\'on:}\\
 El uso de este tipo de plataformas coolaborativas nos permiten evaluar constantemente el trabajo de cada grupo como as\'i tambien cooperar aportando conocimientos
 y mejorando los contenidos. Entonces, para la entrega final el contenido es el resultado de las sumas de las partes, por lo tanto la evaluaci\'on es grupal
 pero puede estar dada individualmente tambi\'en si el docente lo requiere. Podemos evaluar en cada entrega, tambi\'en podemos evaluar la calidad de entrega de 
 cada grupo y su predisposici\'on a ayudar a los dem\'as.\\
 Al inicio, en la selecci\'on del tema debemos clarificar el resultado esperado.
 \end{enumerate}

\subsection{github}
\title{Licencias}
Apache License 2.0 
GNU General Public License v2.0
MIT License
Artistic License 2.0
BSD 2-clause "Simplefied" License
BSD 3-clause "New" or "Revised" License
Creative Commons Zero v1.0 Universal
Eclipse Public License 1.0
GNU Alfero General Public v3.0
GNU General Public License v3.0
GNU Lesser General Public License v2.1
GNU Lesser General Public License v3.0
ISC License
Mozilla Public License 2.0
The Unlicense
 
\section{Claves}
\begin{center}
 \textit{ Donde, cuando y como interviene el docente?}
\end{center} 
El docente interviene durante todo el per\'iodo de desarrollo. Estas son las actividades propuestas:\\
\begin{enumerate}
 \item Monitoreo constante sobre el desarrollo. Sobre todo controlar si las bases y contenidos de las investigaciones realizadas por los grupos de alumnos son
 correctas.
 \item Proponer debates, ampliar contenidos, ejemplificar, son tareas en las que el docente puede intervenir.
 \item Resolver y evacuar dudas que en oportunidades los grupos pueden pedir apoyo al docente.
 \item Evaluaciones parciales sobre entregas o demostraciones, esto permite tener un mejor entendimiento y llegar a una conclusi\'on bien formada al final del
 trabajo.
\end{enumerate}

\section{Beneficios}
Podemos listar los siguientes beneficios:\\
\begin{itemize}
 \item Participaci\'on de cada uno de los alumnos.
 \item Dar tiempo y espacio a la creatividad.
 \item Cooperaci\'on entre los equipos.
 \item Compromiso entre todos para lograr un resultado com\'un.
 \item Fomentar la libre expresi\'on y libertad a la hora de elegir como llevar a cabo el trabajo.
 \item Evaluaci\'on cruzada entre los integrantes de los equipos.
 \item Correcciones tempranas.
\end{itemize}

\section{Conclusi\'on y Resultados}
La aplicaci\'on de esta herramienta o m\'etodo de ense\~nanza tiene algunas ventajas particulares como; el alumno siente mayor ineter\'es en los temas ya que
encuentra la libertad de elecci\'on a trabajar en sobre algo que \'el prefiere, la cooperaci\'on conjunta, el desarrollo sin tiempos y autoevaluaci\'on son las 
m\'as destacadas.\\
Ahora bien, hablemos de los resultados, el docente debe asumir su rol y liderar las actividades como; revisi\'on de contenidos, comunicaci\'on entre grupos, 
entregas parciales y dem\'as. De este modo llegar de la mejor manera posible al final del trabajo.



