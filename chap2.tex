\chapter{Respetar al tiempo}
\emph{
  \begin{center}
    En educaci\'on el tiempo es uno de los recursos m\'as preciados. El tiempo nos rodea, se mueve con nosotros, nos acompa\~na. El tiempo es donde cualquier
    actividad educativa empieza y termina, incluso cuando menos tiempo tenemos es cuando m\'as valorables es. Entonces, como no saber respetar el tiempo, la 
    planificaci\'on, componer los calendarios .. dar el tiempo.\\
    En los negocios dicen que el tiempo es oro para los empresarios. En la educaci\'on el tiempo tendr\'ia que ser el futuro de los estudiantes, entonces, 
    cuidemos, respetemos y hagamos honor al futuro de los estudiantes.
  \end{center}}
  
\section{Planificación acad\'ecima}
Planificar los contenidos del dictado de una materia es igual a la estructura de un \'arbol; primero las raices firmes en la tierra, luego el tronco y despu\'es
las ramas con sus respectivas hojas. Todo tiene que estar conectado con el tema central de la mat\'eria\\
  
\section{Relog - Agenda - Calendario}
En toda planificaci\'on acad\'emica existen calendarios que seguir, programas que cumplir y ciertos contenidos que debatir.\\
Es importante mantener la constante y abierta comunicaci\'on del ritmo de la materia con toda la audiencia estudiantil, esto quiere decir, los individuos 
necesitan saber:
\begin{itemize}
 \item ¿Donde estamos durante el transcurso del dictado?
 \item ¿Cu\'anto tiempo nos queda?
 \item ¿Que contenidos nos faltan?
 \item ¿C\'omo es el progreso actual y su calidad?
 \item ¿Cuales son los temas siguientes?
\end{itemize}
Estos tipos de preguntas tienen que ser evacuadas de manera r\'apida y sencillas por los estudiantes. Los docentes son los encargados de proveer esta
informaci\'on. Ahora bien, la pregunta es: \textit{¿C\'omo ponemos esta informaci\'on al alcance de los alumnos?}.\\
Para esto podemos implementar una poderosa herramienta conocida como \textbf{Facilitaci\'on gr\'afica}.
\subsection{Facilitaci\'on gr\'afica}
La facilitaci\'on gr\'afica ayuda a la gente a entender situaciones complejas, identificar sus elementos claves, evidenciar sus relaciones y tomar mejores 
decisiones, favoreciendo as\'i el trabajo colaborativo.\cite{fg}\\
\subsection{Implementac\'on}
Ahora que conocemos un poco m\'as en detalle el poder de la facilitaci\'on gr\'afica vamos a particularizarlo como herramienta slow.\\
Es importante recalcar que la receta que describiremos a continuaci\'on puede ser aplicada de m\'ultiples y variados formatos.\\
Para la construcci\'on necesitamos de tres variables esenciales:
\begin{enumerate}
 \item \textbf{Alcance}
 El alcance es nuestro esqueleto, nuestro \'arbol completo (siguiendo el ejemplo del principio de \'este cap\'itulo), es donde est\'a todo el contenido del 
 dictado de la materia. Nos sirve para tener un claro entendimiento sobre que es lo que realmente abarca la materia en questi\'on.\\
 Los modelos de esqueleto o representaci\'on del alcance puede estar dise\~nado seg\'un la tem\'atica a estudiar.
 \item \textbf{Medici\'on del tiempo}
 Hablamos de darle tiempo al tiempo, pero los plazos en educaci\'on son finitos para el dictado de una materia, por ello es vital definir el principio y fin de 
 cada tema, de cada tronco, tallo y hojas del \'arbol. El segundo paso nos lleva a dividir nuestro alcance en \textit{n} lotes o m\'odulos los cuales tienen
 como datos de entrada la variable n\'umero tres. Es importante que el tama\~no de los lotes coincida con el tiempo de desarrollo.
 \item \textbf{Control de contenidos}
 Ya tenemos nuestro esqueleto dividido con sus respectivos lotes. Ahora es necesario completar en manera conjunta, profesor y alumnos, mediante un control de 
 contenidos cada uno de los lotes. Esto refleja el porcentaje de completaci\'on y avance de la materia.\\
 A esta variable es posible agregale un factor clave en educaci\'on, \textbf{la calidad}. Para esto necesitamos tambi\'en evaluar cada progreso de los lotes o
 m\'odulos. Como conclusi\'on, completar un lote puede estar representado con m\'as de dos variables las cuales deben ser f\'acilmentes identificables.
\end{enumerate}

\subsection{Ejemplos de implementaci\'on}


\section{Definir los tipos de tiempos}

\section{Beneficios}
El objetivo final de la implementaci\'on de esta herramienta es ayudar a los docentes y estudiantes a tomar mejores decisiones sobre el tiempo que est\'a
transcurriendo. Si de forma colaborativa podemos plasmar los resultados y estado actual logramos obtener:
\begin{itemize}
 \item Mejor entendimiento del contexto del curso.
 \item Tomar decisiones sobre el tiempo restante.
 \item Ver reflejado el presente para mejorar el futuro.
\end{itemize}


\section{Aplicaci\'on}