\chapter{Ense\~nar a pensar}
\emph{
  \begin{center}
    Educar no es f\'acil y menos a\'un lo es enseñar a pensar. Ya que ambas cosas suponen esfuerzo y dedicaci\'on.\cite{en} \\
    Obedecer no es educativo, al contrario de lo que estamos acostumbrados a escuchar, obedecer no educa, no ense\~na, s\'olo nos sirve para generar sumisi\'on 
    y asegurarnos que todo estar\'a bajo nuestro control cuando consigamos que nos obedezcan nuestros alumnos.
  \end{center}
}    
\section{Las estructuras del proceso creativo}
\'Este cap\'itulo est\'a enfocado a ese\~nar o ayudar a pensar, pero antes es necesario definir algunos conceptos.\\
\textit{¿Qu\'e es realmente ense\~nar a pensar?, ¿se puede ense\~nar a pensar?}\\
Cuando hablamos de ense\~nar a pensar nos referimos a pulir el proceso creativo, a inspirar y presentar trabajos distintos, donde los alumnos sean capaces de
manejar alternativas a lo tradicional. Para eso los docentes deben ser gu\'ias o falicitadores mediante esta herramienta.\\
Se puede extrapolar el desarrollo convencional a algo distinto e innovador, libre y creativo, culto y did\'actico.

\section{Ofrecerle la oportunidad de hacer sus propios caminos}
No darle las cosas hechas, solucionadas o terminadas, sino que, desde el apoyo y el acompañamiento, permitirle a los estudiantes hacer, a\'un a 
riesgo de que se equivoque y aunque ello suponga rectificar m\'as tarde.
\subsection{Dar tiempo para crear}
El mecanismo de crear generalmente construye o busca la soluci\'on a un problema. Creamos tambi\'en para aprender nuevas cosas. Ahora bien, para crar cualquier 
tipo de cosa necesitamos par\'ametros de entradas los cuales seguir, por ejemplo, para crear un edificio necesito saber; tipo, altura, tama\~no, etc, o para
construir una soluci\'on a un problema matem\'atico es necesario conocer; la operaci\'on, el resultado esperado, otros m\'etodos aplicables, etc. \\
Dar tiempo a la creaci\'on es: 
%grafico sobre dise\~no .. revision .. feedback y evaluacion de resultados.
\subsection{¿Cuando aplicamos esta herramienta?}
¿Qu\'e necesitamos de nuestros alumnos como creadores?. La respuesta es simple, que sean capaces de solucionar un problema o dejarles una ense\~nanza.\\
%crear cartas con cada uno de estos pasos con checklist para saber si estan listas y continuar a la siguiente fase
%las cartas con como las hojas de desing thinking.
El desarrollo y real aplicaci\'on de esta herramienta puede ser en trabajos de campos, trabajos pr\'acticos, evaluaciones de desempe\~no, recuperaciones, 
monograf\'ias, etc. Para ello la siguiente lista de pasos:\\
\begin{enumerate}
 \item \textbf{Formular el problema:}
 El rol del docente en este paso es vital para el buen desarrollo y \'exito de esta herramienta. Es necesario que el docente sea observador de que es realmente
 lo esperado de los trabajos, es decir, lo entregable o resultado final. Despu\'es, formular las descripciones consiste en plantear las metas a lograr por el o 
 los alumnos, pautas las condiciones m\'inimas requeridas y la definici\'on de cuando estar\'a listo.\\
 Por ejemplo; docentes de programaci\'on, el resultado final de un cierto trabajo es la construcci\'on de un determinado software y su descripci\'on son los 
 requerimientos del sistema. O un docente de f\'isica, lograr la formulaci\'on y resoluci\'on de conflictos en planos el\'ectricos son metas y la descripci\'on
 tiene que estar dada por las variables intervinientes y condiciones externas.\\ 
 Los docentes deben ser creativos a la hora de definir las problem\'aticas, porque para ayudar o ense\~nar a pensar es condici\'on necesaria que el medio o 
 dise\~no que la soluci\'on no sea \'unica. Por ejemplo, la f\'ormula de velocidad es distancia sobre tiempo, existen muy pocos caminos para llegar a ese 
 resultados y todos son conocidos. 
 
 \item \textbf{Observar y dar feedback sobre dise\~nos:}
 El desaf\'io de la soluci\'on pasa por el dise\~no, cuanto m\'as preciso, determin\'istico, acotado y simple sea es mejor. Entonces, como paso inmediato a 
 formular el problema y previamente comprobar que los alumnos entiendan el problema a solucionar, podemos dar el segundo paso que es evaluar los dise\~nos 
 presentados dando feedback o corroborando criterios para que funcione bien, como ser:
 \begin{itemize}
  \item Que el dise\~no sea inspirador y real, es decir, que pueda ser llevado a cabo.
  \item Que en cualquier parte del proceso acepte resultados cambiantes y presente alternativas a respuestas negativas.
  \item Que parta de la imaginaci\'on y no de un dise\~no que ya exista.
  \item Que no presente ning\'un supuesto ni resultado por adelantado.
  \item Que presente \'areas inesperadas sin explorar para trabajos posteriores.
 \end{itemize}

 \item \textbf{Desarrollar y prototipar:}
 Para el desarrollo y futuro \'exito del proyecto es necesario tener en cuenta los siguientes puntos para luego pasar las silumaciones y evaluaciones:
 Supongamos como ejemplo que nuestro problema a solucionar es la desnutrici\'on infantil acotado en la escuela del barrio. En el punto anterior se dise\~no
 como soluci\'on la creaci\'on de una comida gustosa, sana, econ\'omica y nutritiva como lo es la ensalada de frutas.
 \begin{itemize}
  \item \textit{Hip\'otesis afianzada al dise\~no:}\\
  Partimos de la hip\'otesis que a la mayor\'ia de los ni\~nos les gutan las frutas, hay frutas locales econ\'omicas y es de f\'acil preparaci\'on. Entonces 
  hay que validar la hip\'otesis en la escuela y mercados cercanos.
  \item Supuestos riesgosos:\\
  Si validaciones como; a los ni\~nos no les gustan las frutas o las frutas locales son realmente costosas, el dise\~no propuesto no es v\'alido y tenemos que
  retornar al punto anterior.
  \item \textit{Manos a la obra:}\\
  Tenemos que preparar la mayor cantidad de ensaladas de frutas, con la mayor proteina posible y al menos costo. \'Esta etapa es la m\'as pura del desarrollo.
  \item \textit{Prototipar:}\\
  Este paso consiste en evaluar el desarrollo mediante la presentaci\'on de prototipos, pueden ser; gr\'aficos, maquetas, artefactos, objetos, idealmente algo 
  con que interactuar y experimentar, cualquiera de estos medios que respondan a preguntas sobre el desarrollo y acerquen a la soluci\'on final.\\
  ¿Porqu\'e prototipar? 
  \begin{itemize}
   \item Para poder construir e invertar al mismo tiempo.
   \item Para comunicar y presentar entregables.
   \item Para empezar conversaciones e intercambiar ideas.
   \item Para cometer errores en tiempo de dise\~no y desarrollo no tan avanzados, entonces fallar temprano es m\'as barato.
   \item Para analizar alternativas y evaluar robustez.
  \end{itemize}
  Siguiendo el ejemplo, podemos presentar gr\'aficos estad\'isticos sobre costos, variables energ\'eticas y consumos estimados o prototipar una ensalada de 
  frutas.
 \end{itemize}
 
 \item \textbf{Recabar informaci\'on sobre como funciona:}
  Hasta esta instancia tenemos; el compromiso del o los alumnos para afrontar el desaf\'io del proyecto, posibles soluciones al problema mediante el dise\~no y
  las bases de construcci\'on reflejadas en prototipos entregables. En este punto nos encargamos del an\'alisis y evaluaci\'on del funcionamiento, para ello 
  debemos contar con m\'etodos de simulaci\'on o ambientes de prueba:
  \begin{itemize}
   \item \textbf{Simulaci\'on y validaci\'on:}
   Analizar los resultados de cada uno de los pasos en la etapa de desarrollo:
   \begin{itemize}
    \item ¿Cu\'al es el problema? Definirlo en menos de 3 minutos.
    \item Validar la o las soluciones.
    \item Identificar los riesgos y modos de control.
    \item Viabilidad.
   \end{itemize}

   \item \textbf{CheckList:}
   Para seguir agregando actividades o entrenamiento al pensamiento sobre el trabajo que ya esta en pleno desarrollo, esta lista de preguntas ayudar\'an
   a seguir aportando valor:\\
   \begin{itemize}
    \item ¿Qu\'e particularidad o nueva funcionalidad le agregar\'ias al proyecto?.
    \item ¿C\'omo pensas que se puede llegar a mejorar?.
    \item Si tendr\'ian que trabajar desde cero, ¿Qu\'e cambiar\'ian?.
    \item ¿Qu\'e experimento o pruebas piensan que este proyecto no superar\'ia?.
    \item ¿C\'omo expandir esta soluci\'on a otros \'ambitos?.
   \end{itemize}
  
  \end{itemize}
  Como punto importante en este objetivo general que desarrollamos en \textsl{ense\~nar a pensar} y como es habitual en el mundo actual, los requerimientos 
  cambian, entonces necesitamos volver a dise\~nar, volver a comprobar si lo propuesto cubrir\'a lo nuevo y mantiene lo anterior.\\
  Para esto, definimos nuevos retos sobre lo ya construido como parte de la especificaci\'on anterior y largamos a rodar de nuevo los pasos anteriores.
\end{enumerate}
\subsection{Ciclo repetitivo}
repetir muchas veces los pasos anterioes

\section{Estimulaci\'on}
superaci\'on y crecimiento
razonamiento, resoluci\'on de problemas y toma de desiciones.
\section{Beneficios}
%Para la tesis http://hksec.hk/sites/default/files/attachment/805/javelin-experimentboard.pdf