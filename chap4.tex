\chapter{Ense\~nar a pensar}
\emph{
  \begin{center}
    Educar no es f\'acil y menos a\'un lo es enseñar a pensar. Ya que ambas cosas suponen esfuerzo y dedicaci\'on.\cite{en} \\
    Obedecer no es educativo, al contrario de lo que estamos acostumbrados a escuchar, obedecer no educa, no ense\~na, s\'olo nos sirve para generar sumisi\'on 
    y asegurarnos que todo estar\'a bajo nuestro control cuando consigamos que nos obedezcan nuestros alumnos.
  \end{center}
}    
\section{Definir los tipos de tiempos}
\section{Ofrecerle la oportunidad de hacer sus propio camino}
es decir, no darle las cosas hechas, solucionadas o terminadas, sino que, desde el apoyo y el acompañamiento, permitirle a los estudiantes hacer, a\'un a 
riesgo de que se equivoque y aunque ello suponga rectificar m\'as tarde.
\subsection{Dar tiempo para crear}
desing thinking
\section{Estimulaci\'on}
superaci\'on y crecimiento
razonamiento, resoluci\'on de problemas y toma de desiciones.
\section{Beneficios}