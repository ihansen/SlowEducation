% $Log: abstract.tex,v $
% Revision 1.1  93/05/14  14:56:25  starflt
% Initial revision
% 
% Revision 1.1  90/05/04  10:41:01  lwvanels
% Initial revision
% 
%
%% The text of your abstract and nothing else (other than comments) goes here.
%% It will be single-spaced and the rest of the text that is supposed to go on
%% the abstract page will be generated by the abstractpage environment.  This
%% file should be \input (not \include 'd) from cover.tex.
El movimiento Slow se origin\'o a mediados de 1980 en Italia. Actualmente posee m\'ultiples y variadas aplicaciones, SlowEducation es una de ellas.\\
El desarrollo de esta tesis tiene como objetivo implementar m\'etodos de Slow + SlowEducation a niveles universitarios, con ejemplos pr\'acticos de 
aplicaci\'on de campo para Ing. en Sistemas de Informaci\'on.\\
Los resultados finales de experimentar m\'etodos Slow en educaci\'on universitaria son los mismos, varian las carreras de grado dependiendo de herramientas 
y tecnolog\'ias aplicadas.\\
La educaci\'on lenta una la relaci\'on entre el estudiante, el profesor y la Universidad, conforman as\'i entre estos tres elementos una sociedad de 
aprendizaje. El objetivo de aplicar estos m\'etodos est\'a orientado a afianzar las competencias b\'asicas adquiridas por el alumnado y la transferibilidad 
de \'estas a la vida cotidiana profesional/laboral.


