% $Log: abstract.tex,v $
% Revision 1.1  93/05/14  14:56:25  starflt
% Initial revision
% 
% Revision 1.1  90/05/04  10:41:01  lwvanels
% Initial revision
% 
%
%% The text of your abstract and nothing else (other than comments) goes here.
%% It will be single-spaced and the rest of the text that is supposed to go on
%% the abstract page will be generated by the abstractpage environment.  This
%% file should be \input (not \include 'd) from cover.tex.
El movimiento Slow se origin\'o a mediados de 1980 en Italia. Actualmente posee m\'ultiples y variadas aplicaciones, SlowEducation es una de ellas.\\
El objetivo en la investigaci\'on y desarrollo de esta tesis es implementar m\'etodos de Slow + SlowEducation a nivel de educaci\'on universitaria, 
con ejemplos pr\'acticos de aplicaci\'on en el campo de carreras de grado.\\
Los resultados finales de experimentar m\'etodos Slow en educaci\'on deber\'ian ser los mismos, a pesar de las herramientas y tecnolog\'ias aplicadas en las
diferentes especialidades.\\
SlowEducation o educaci\'on lenta es una relaci\'on entre el estudiante, el profesor y la Universidad, conformando as\'i entre estos tres elementos una sociedad 
de aprendizaje. El prop\'osito de aplicar estos m\'etodos est\'a orientado a afianzar las competencias b\'asicas adquiridas por el alumnado y la transferibilidad 
de \'estas a la vida cotidiana profesional/laboral.


